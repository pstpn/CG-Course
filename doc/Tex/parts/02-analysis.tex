\chapter{Аналитический раздел}

\section{Описание предметной области}

Компьютерная графика изначально зародилась как эффективное и мощное средство связи между человеком и вычислительной машиной.
Использование графической формы представления информации, организация диалога между человеком и компьютером с использованием визуальных образов позволили существенно увеличить скорость обработки информации человеком, что привело к повышению эффективности исследований и разработок в самых различных областях науки и техники.
Одним из направлений компьютерной графики является визуализация звуковых волн \cite{baseGraph1, baseGraph2}.

Звук~--- физическое явление, представляющее собой распространяющиеся в виде упругих волн механические колебания в твердой, жидкой или газообразной среде.

Изучение природы звуковых волн является важной частью на пути к пониманию и представлению более сложных физических процессов, что достигается путем применения методов визуализации.

\clearpage

\section{Формализация объектов сцены}

Объектами сцены будут являться:
\begin{enumerate}
	\item замкнутое пространство~--- помещение прямоугольной формы, состоящее из четырех стен, пола и потолка и имеющее константный размер, которое не поглощает энергию волны;
	\item препятствия~--- объекты кубической формы, полностью поглощающие энергию волны;
	\item точечный источник звуковых волн~--- материальная точка в пространстве, характеризующаяся следующими свойствами:
	\begin{itemize}
		\item источник звука излучает звуковые волны во все стороны равномерно;
		\item источник не имеет предпочтительного направления для распространения звука;
		\item звук распространяется от источника радиально, образуя сферическую волну.
	\end{itemize}
	\item источник света~--- материальная точка, задающая параметры освещенности объектов сцены;
	\item камера~--- материальная точка в пространстве, которая определяет обзор и параметры визуализации.
\end{enumerate}

\section{Методы описания моделей на сцене}

В компьютерной графике выделяют три основных вида описания моделей на сцене \cite{models}:
\begin{enumerate}
	\item каркасные~--- дают представление о поверхности объекта, но описывают его только дискретными элементами каркаса (точки или линии);
	\item поверхностные~--- также дают представление о поверхности объекта, но несут информацию обо всех точках, принадлежащих этой поверхности;
	\item твердотельные~--- объекты в виде сплошных тел, т.~е. в виде сочетания всех точек занимаемого моделью объема.
\end{enumerate}

В данном курсовом проекте выдвинуты следующие критерии к методам описания моделей на сцене \cite{compareModels1}:
\begin{enumerate}
	\item сложность обработки;
	\item количество информации о модели.
\end{enumerate}

\section*{Вывод}

В таблице \ref{table:compareModels} приведены результаты сравнения методов описания моделей на сцене, исходя из выдвинутых критериев \cite{compareModels2}.
\begin{table}[h!]
	\begin{center}
		\caption{\label{table:compareModels} Сравнительная таблица для методов описания объектов на сцене}
		\begin{tabular}{|p{100pt}|p{100pt}|p{120pt}|p{120pt}|}
			\hline
			~ & Каркасная & Поверхностная & Твердотельная \\ \hline
			Сложность\newline обработки & $O(k \cdot n)$ & $O(n^2)$ & $O(n^3)$ \\ \hline
			Количество\newline информации & Только вершины модели & Каркасная модель, дополненная информацией о всех точках поверхности & Поверхностная модель, дополненная информацией о внутренних точках \\ \hline
		\end{tabular}
	\end{center}
\end{table}

В качестве реализуемых методов были выбраны поверхностный способ для описания препятствий сцены, так как он обладает оптимальной сложностью обработки $O(n^2)$ и при этом предоставляет всю необходимую информацию для решения поставленных задач, и каркасный способ для описания звуковой волны, которая составляет значительную часть объектов сцены, для достижения наименьшей сложности работы алгоритмов.

\clearpage

\section{Выбор алгоритма удаления невидимых линий и поверхностей}

Для построения реалистичных изображений необходимо предусмотреть возможность удаления невидимых линий и поверхностей.

Выделяют два основных вида таких алгоритмов \cite{invisible1}:
\begin{enumerate}
	\item алгоритмы, работающие в пространстве объекта: использующий Z-буфер, Варнока;
	\item алгоритмы, работающие в пространстве изображения: Робертса.
\end{enumerate}

В данном курсовом проекте выдвинуты следующие критерии к алгоритмам удаления невидимых линий и поверхностей \cite{compareInvisible1}:
\begin{enumerate}
	\item система координат, с которой работают алгоритмы;
	\item объем вычислений.
\end{enumerate}

В таблице \ref{table:compareInvisible} приведены результаты сравнения алгоритмов, работающих в пространстве объекта и в пространстве изображения \cite{compareInvisible1}
\begin{table}[h!]
	\begin{center}
		\caption{\label{table:compareInvisible} Сравнительная таблица для алгоритмов, работающих в пространстве объекта и в пространстве изображения}
		\begin{tabular}{|p{70pt}|p{190pt}|p{190pt}|}
			\hline
			~ & Алгоритмы, работающие в пространстве объекта & Алгоритмы, работающие в пространстве изображения \\ \hline
			Система координат & Алгоритмы работают с физической системой координат & Алгоритмы работают с экранной системой координат \\ \hline
			Объем\newline вычислений & Растет, как квадрат числа\newline объектов $n$: $O(n^2)$ & Растет, как число объектов $n$, умноженное на число пикселей $k$: $O(n \cdot k)$ \\ \hline
		\end{tabular}
	\end{center}
\end{table}

Далее рассматриваются только алгоритмы, работающие в пространстве изображения, так как они обладают линейной сложностью $O(n)$ (при фиксированном количестве пикселей на экране), а также работают с экранной системой координат.

\subsection{Алгоритм, использующий Z-буфер}

Алгоритм был предложен Эдом Кэтмулом и представляет собой обобщение буфера кадра.
Обычный буфер кадра хранит в пространстве изображения коды цвета для каждого пикселя.

Идея алгоритма удаления поверхностей с Z-буфером состоит в том, чтобы для каждого пикселя дополнительно хранить величину глубины или координату Z.
Когда очередной пиксель заносится в буфер кадра, происходит сравнение значения его Z-координаты с координатой Z пикселя, который уже имеется в буфере.
Атрибуты нового пикселя и его Z-координата заносятся в буфер, если он ближе к наблюдателю, т.~е. если Z-координата нового пикселя больше, чем координата старого.

Главное преимущество алгоритма заключается в его простоте, однако для его реализации требуется большой объем памяти. 
Кроме решения общей задачи удаления невидимых линий и поверхностей он позволяет достаточно просто вычислять изображение сечения трехмерного объема плоскостью с произвольной координатой Z{сеч}, что предоставляет возможность обрабатывать сцену любой сложности \cite{base, baseOpt}.

\subsection{Алгоритм Варнока}

Алгоритм работает в пространстве изображения и анализирует область на экране дисплея (окно) на наличие в них видимых элементов:
\begin{itemize}
	\item если в окне нет изображения, то оно просто закрашивается фоном;
	\item если в окне имеется элемент, то проверяется, достаточно ли он прост для визуализации;
	\item если объект сложный, то окно разбивается на более мелкие, для каждого из которых выполняется тест на отсутствие и/или простоту изображения.
\end{itemize}
Рекурсивный процесс разбиения может продолжаться до тех пор пока не будет достигнут предел разрешения экрана \cite{invisible1}. 

Алгоритм может быть реализован в двух вариантах в зависимости от решаемой задачи:
\begin{enumerate}
	\item удаление невидимых линий, в результате чего получается контурное изображение элементов сцены;
	\item удаление невидимых поверхностей.
\end{enumerate}

Реализация двух подходов различается в определении цвета областей, которые являются частями изображения каких-либо граней. 
Если область содержит пиксели, относящиеся к одной грани, и не содержит границ полигонов, то при удалении невидимых поверхностей визуализируемое изображение принимает значение цвета этой грани.
Для случая удаления невидимых линий такая область считается <<пустой>>, то есть не содержащей элементов изображения \cite{base}.

\section*{Вывод}

В качестве реализуемого алгоритма удаления невидимых линий и поверхностей был выбран алгоритм, использующий Z-буфер в силу его простоты и универсальности (в виде обработки сцен любой сложности).

\clearpage

\section{Выбор модели освещения}

Для корректного наблюдения за визуализацией распространения волн существует необходимость в выборе модели освещения.

Выделяют две основные модели освещения \cite{baseLight}:
\begin{enumerate}
	\item модель Ламберта;
	\item модель Фонга.
\end{enumerate}

\subsection{Модель Ламберта}

В общем виде модель освещения Ламберта состоит из суммы фоновой и диффузной компонент:
\begin{equation}\label{equ:lambert}
	I = I_{a} + I_{d} = m_{a} \cdot L_{a} + m_{d} \cdot k_{d} \cdot L_{d}
\end{equation}
где, $I_{a}$~--- интенсивность фоновой составляющей,
\newline $I_{d}$~--- интенсивность диффузной составляющей.

Модель Ламберта является одной из самых простых моделей освещения.
Данная модель очень часто используется как часть других моделей, поскольку практически в любой другой модели освещения можно выделить диффузную составляющую.
Данная модель может быть очень удобна для анализа свойств других моделей.
Она является существенной частью модели Фонга \cite{baseLight}.

\subsection{Модель Фонга}

В 1975 Фонг предложил модель освещения достаточно гладких поверхностей.
Эта модель давно стала классикой и до сих пор остается самой популярной в компьютерной графике.
В общем виде модель освещения Фонга состоит из суммы фоновой, диффузной и зеркальной составляющей и представляется формулой \ref{equ:phong} \cite{baseLight}:
\begin{equation}
	\label{equ:phong}
	I = I_{a} + I_{d} + I_{s}
\end{equation}
где, $I_{a}$~--- интенсивность фоновой составляющей,
\newline $I_{d}$~--- интенсивность диффузной составляющей,
\newline $I_{d}$~--- интенсивность зеркальной составляющей.

\section*{Вывод}

Исходя из потребности в качественном и детальном наблюдении за визуализацией распространения волн, была выбрана модель освещения Фонга, являющаяся более сложной и классической по сравнению с моделью Ламберта.

\section{Выбор метода закраски}

Для идентификации и визуально корректного отображения объектов сцены необходимо выбрать алгоритм закраски.
Выделяют три основных вида методов закраски \cite{draw1}:
\begin{enumerate}
	\item простая закраска;
	\item закраска по Гуро;
	\item закраска по Фонгу.
\end{enumerate}
Далее будут рассмотрены только метод простой закраски и метод закраски по Гуро, так как метод закраски по Фонгу требует больших вычислительных затрат, которые не являются необходимыми \cite{base}.

\subsection{Метод простой закраски}

Алгоритм простой закраски вызывает расчет по модели освещения только 1 раз, в одной контрольной точке, которая может быть как вершиной примитива, так и его центром.
Полученный таким образом цвет применяется ко всем пикселям примитива.

Этот алгоритм используется в том случае, когда важно не качество изображения, а производительность и относительно небольшие вычислительные затраты \cite{baseLight}.

\clearpage

\subsection{Метод закраски по Гуро}

Метод Гуро~--- закраска, согласно которой цвет примитива рассчитывается лишь в вершинах, а затем линейно интерполируется по его поверхности, что значительно снижает вычислительные затраты.

Закраска граней по методу Гуро осуществляется в четыре этапа \cite{draw1}:
\begin{enumerate}
	\item вычисление нормали к каждой грани;
	\item определение нормали в вершинах многогранника путем усреднения нормали по всем полигональным граням, которым принадлежит вершина;
	\item вычисление значения интенсивности освещения в вершинах при помощи нормали в вершинах;
	\item закрашивание каждого многоугольника путем линейной интерполяции значений интенсивности в вершинах.
\end{enumerate}

\section*{Вывод}

Исходя из потребности только в идентификации объектов сцены и быстродействии, в качестве метода закраски был выбран метод простой закраски.

\section{Вывод}

В данном разделе были формализованы объекты сцены, а также рассмотрены методы описания моделей на сцене.
Кроме того, были рассмотрены алгоритмы удаления невидимых линий и поверхностей, модели освещения и методы закраски.

Были выбраны следующие алгоритмы, модели и методы:
\begin{enumerate}
	\item в качестве алгоритма удаления невидимых линий и поверхностей~--- алгоритм, использующий Z-буфер;
	\item в качестве модели освещения~--- модель освещения Фонга;
	\item в качестве метода закраски~--- метод простой закраски.
\end{enumerate}